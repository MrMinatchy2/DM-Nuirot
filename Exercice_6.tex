\hypertarget{Exercice_6}{
\Huge{\begin{center}Exercice 6\end{center} \leavevmode\newline }}

\hypertarget{resolution}{%
\LARGE{Résolution}\label{resolution}}
\newline
\newline
Vérifions si les propriétés sont bien respecté.
\newline
\newline
Réflexivité: Soit $A \in E$ nous avons bel et bien $ARA$ car $A = A$.
\newline
Symétrie: Par définition nous avons bel et bien si $A=B$, $B=A$ ou si $A=\overline{B}$, $B=\overline{A}$
\newline
Transitivité:  Nous avons donc soit $A, B, C \in E$ avec $ARB$ et $BRC$ donc:
\newline
\newline
Si $A=B$ et $B=C$ nous avons bel et bien $A=C$.
\newline
Si $A=B$ et $B=\overline{C}$ nous avons $A=B$ et donc $A=\overline{C}$.
\newline
Si $A=\overline{B}$ et $B=C$ nous avons $B=C$ et donc $A=\overline{C}$.
\newline
Si $A=\overline{B}$ et $B=\overline{C}$ nous avons $C=\overline{B}$ et donc $A=C$.
\newline
\newline
Donc nous avons bel et bien une relation d'équivalence.
