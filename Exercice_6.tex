\hypertarget{Exercice_3}{
\Huge{\begin{center}Exercice 3\end{center} \leavevmode\newline }}

\hypertarget{resolution}{%
\LARGE{Résolution}\label{resolution}}
\newline
\newline
Vérifions si les propriétés sont bien respecté.
\newline
\newline
Réflexivité: Soit $A \in E$ nous avons belle et bien $ARA$ car $A = A$.
\newline
Symétrie: Par définition nous avons belle et bien si $A=B$, $B=A$ ou si $A=\overline{B}$, $B=\overline{A}$
\newline
Transitivité:  Nous avons donc soit $A, B, C \in E$ avec $ARB et BRC$ donc:
\newline
Si $A=B et B=C$ nous avons belle et bien $A=C$.
Si $A=B et B=\overline{C}$ nous avons $A=B$ et donc $A=\overline{C}$.
Si $A=\overline{B} et B=C$ nous avons $B=C$ et donc $A=\overline{C}$.
Si $A=\overline{B} et B=\overline{C}$ nous avons $C=\overline{B}$ et donc $A=C$.
\newline
\newline
Donc nous avons belle et bien une relation d'équivalence
