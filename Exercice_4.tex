\hypertarget{Exercice_3}{
\Huge{\begin{center}Exercice 3\end{center} \leavevmode\newline }}

\hypertarget{resolution}{%
\LARGE{Résolution}\label{resolution}}
\newline
\newline
1. $\rceil p = (\forall t\in \mathbb{R}, \exists x \in \mathbb{R}, t \leq f(x))$
\newline
\newline
2. Si $f(x) = \cos (x)$ cette assertion est vrai car cette fonction est majorée.
\newline
Si $f(x) = x^2$ cette assertion est fausse car cette fonction n'est pas majorée.
\newline
\newline
3.1
\newline
\newline
a. Cette proposition est équivalente à p.
\newline
\newline
b. Cette proposition est fausse car l'ensemble $\mathbb{R}$ n'est pas minoré.
\newline
\newline
c. Cette proposition n'est pas toujours vrai car vrai si on prend $f(x) = x-1$ et fausse si on prend $f(x) = x^2$ et $t=-1$.
\newline
\newline
d. Cette proposition est toujours vrai car l'ensemble $\mathbb{R}$ admet toujours un élément plus grand qu'un autre.
