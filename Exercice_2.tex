\hypertarget{Exercice_2}{
\Huge{\begin{center}Exercice 2\end{center} \leavevmode\newline }}


\hypertarget{resolution}{%
\LARGE{Résolution}\label{resolution}}
\newline
\newline
Notons p = ”il pleut ”, q =” Abel se promène avec un parapluie” et r = ”Béatrice se promène avec un parapluie”.
\newline
on sait que: $p\Rightarrow q$ et $p \Rightarrow r$.
\newline
\newline
1. On ne peut rien conclure car q peut être vrai que p soit vrai ou faux et donc que r soit vrai ou faux.
\newline
\newline
2. $\rceil q \Rightarrow \rceil p \Rightarrow \rceil r$
\newline
\newline
3. $r \Rightarrow p \Rightarrow q$
\newline
\newline
4. $\rceil r \Rightarrow \rceil p$
\newline
\newline
5. $\rceil p \Rightarrow \rceil r$
\newline
\newline
6. $p \Rightarrow q \Rightarrow q$
