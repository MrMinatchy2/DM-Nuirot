\hypertarget{Exercice_7}{
\Huge{\begin{center}Exercice 7\end{center} \leavevmode\newline }}

\hypertarget{resolution}{%
\LARGE{Résolution}\label{resolution}}
\newline
\newline
Vérifions que les propriétés sont respectée.
\newline
\newline
Réflexivité: $pRp = p^1$ pour tout $p \in \mathbb{N}*$.
\newline
Antisymétrie: si $pRq$ et $qRp$ nous avons $p=q^k$ et $q=p^j$ avec $k,j \geq 1$ donc $p=p^{jk}$.Cela implique que $p=1$ donc $q=1$ ou $jk=1$ donc $j=k=1$ et $p=q$.
\newline
\newline
Transitivité: si $pRq$ et $qRp$, nous avons $q=p^k$ et $r=q^j=p^{jk}$, ceci implique $pRr$.
\newline
\newline
Mais cette relation, d'ordre n'est pas total du fait que par exemple on ne puisse pas comparé $2 et 3$.
\newline
\newline
Démontrons par l'absurde soit p un majorant de ${2,3}$
\newline
\newline
